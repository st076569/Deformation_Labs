%%%%%%%%%%%%%%%%%%%%%%%%%%%%%%%%%%%%%%%%%%%%%%%%%%%%%%%%%%%%%%%%%%%%%%%%%%%%%%%%
% Лабораторная работа 5 Микротвердость

% Выполнили             : Баталов Семен, Хайретдинова Диана, 2021.
%%%%%%%%%%%%%%%%%%%%%%%%%%%%%%%%%%%%%%%%%%%%%%%%%%%%%%%%%%%%%%%%%%%%%%%%%%%%%%%%

\documentclass[12pt, a4paper]{article}
\usepackage[left=2cm, right=2cm, top=2.5cm, bottom=2.5cm, nohead]{geometry}
\usepackage{graphicx}
\usepackage[utf8]{inputenc}
\usepackage[english, russian]{babel}
\usepackage{indentfirst}
\usepackage{amsmath}
\usepackage{longtable}
\usepackage{multirow}
\usepackage{array}
\usepackage{rotating}
\usepackage{subcaption}
\graphicspath{{./Figures/}}

\begin{document}
    
    \newcolumntype{M}[1]{>{\centering\arraybackslash}m{#1}}
    \renewcommand{\arraystretch}{1.3}
    
    \begin{center}
        \large{Санкт-Петербургский Государственный Университет} \\
        \large{Saint-Petersburg State University}\\
        \hfill \break
        \hfill \break
        \hfill \break
        \hfill \break
        \hfill \break
        \hfill \break
        \large{ЛАБОРАТОРИЯ ПРОЧНОСТИ МАТЕРИАЛОВ} \\
        \hfill \break
        \hfill \break
        \hfill \break
        \large{\textbf{ОТЧЕТ}} \\
        \large{\textbf{По лабораторной работе 5}} \\
        \large{<<Определение микротвердости>>} \\
        \hfill \break
        \hfill \break
        \hfill \break
        \large{По дисциплине} \\
        \large{<<Лабораторный практикум, лабораторная работа>>} \\
    \end{center}
    
    \hfill \break
    \hfill \break
    \hfill \break
    \hfill \break
    \hfill \break
    \hfill \break
    
    \begin{flushright} 
        \large{Выполнили:} \\
        \hfill \break
        \large{Баталов С. А.} \\
        \large{Хайретдинова Д. Д.} \\
    \end{flushright}
    
    \hfill \break
    \hfill \break
    \hfill \break
    \hfill \break
    \hfill \break
    
    \begin{center} 
        \large{Санкт-Петербург} \\
        \large{2021} \\
    \end{center}
    
    \thispagestyle{empty}
    \newpage
    \sloppy
    
    \section{Цель работы}
    

    
    Цель работы 
    \newpage
    
    \section{Теоретические исследования}
    
	\begin{table}[h]
	\centering
	\begin{tabular}{|M{1cm}|M{2cm}|M{2cm}|M{2cm}| M{2cm}|}
		\hline
		\multirow{2}{*}{ N } & $d_{1}$ & $d_{2}$ & $d_{\text{ср}}$ & $HV$ \\
		\cline{2-5}
		& \multicolumn{3}{c|}{мкм} & -- \\
		\hline
		1 & 50.5 & 50.3 & 50.4 & 146.0 \\
		2 & 49.2 & 46.2 & 47.7 & 163.3 \\
		3 & 48.6 & 48.2 & 48.4 & 158.2 \\
		4 & 43.2 & 42.2 & 42.7 & 203.8 \\
		5 & 49.8 & 46.7 & 48.3 & 159.2 \\
		6 & 49.3 & 49.0 & 49.1 & 153.7 \\
		7 & 46.2 & 45.6 & 45.6 & 176.2 \\
		8 & 48.3 & 47.2 & 47.8 & 162.6 \\
		9 & 45.3 & 44.8 & 45.0 & 182.8 \\
		10 & 47.9 & 47.1 & 47.5 & 164.4 \\
		\hline
		
	\end{tabular}
	\caption{Исследование микротвердости стального образца при нагрузке 2 Н}
	\label{Steel}
	\end{table}
	
		\begin{table}[h]
	\centering
	\begin{tabular}{|M{1cm}|M{2cm}|M{2cm}|M{2cm}|M{2cm}| M{2cm}|}
		\hline
		\multirow{2}{*}{ N } & $d_{1}$ & $d_{2}$ & $d_{\text{ср}}$ & $HV$ \\
		\cline{2-5}
		& \multicolumn{3}{c|}{мкм} & -- \\
		\hline
		1 & 52.1 & 51.1 & 51.6 & 139.2 \\
		2 & 53.2 & 51.5 & 52.4 & 135.1 \\
		3 & 53.6 & 52.0 & 52.8 & 133.2 \\
		4 & 53.1 & 50.8 & 52.0 & 137.4 \\
		5 & 54.4 & 53.3 & 53.8 & 128.0 \\
		6 & 54.0 & 53.3 & 53.6 & 128.9 \\
		7 & 51.6 & 51.1 & 51.4 & 134.5 \\
		8 & 51.6 & 51.1 & 51.4 & 140.5 \\
		9 & 50.7 & 50.1 & 50.4 & 145.9 \\
		10 & 54.3 & 54.0 & 54.2 & 126.4 \\
		\hline
		
	\end{tabular}
	\caption{Исследование микротвердости медного образца при нагрузке 2 Н}
	\label{Copper}
	\end{table}
	
	\begin{table}[h] 
	\centering
	\begin{tabular}{|M{1cm}|M{2cm}|M{2cm}|M{2cm}| M{2cm}|M{2cm}|}
	\hline
	\multirow{2}{*}{N} & $P$ & $d_{1}$ & $d_{2}$ & $d_{\text{ср}}$ & $HV$ \\
	\cline{2-6}
	& Н &\multicolumn{3}{c|}{мкм} & -- \\
	\hline
	1 & 10 & 108.4 & 107.3 & 107.9 & 159.4 \\
	
	2 & 5 & 76.1 & 75.8 & 75.9 & 160.9 \\
	3 & 	3 & 57.3 & 56.9 & 57.1 & 170.4 \\
	4 & 1 & 33.1 & 32.1 & 32.6 & 174.6 \\
	5 & 0.5 & 25.9 & 22.8 & 24.4  & 156.1 \\
	6 & 0.25 & 20.1 & 19.0 & 19.5 & 121.5 \\
	7 & 0.1 & 8.6  & 8.5 & 8.6 & 252.8 \\
	\hline
	\end{tabular}
     \caption{Исследование микротвердости стального образца при разной нагрузки}
     	\label{St_}
     \end{table}
     
     
     
     	\begin{table}[h] 
	\centering
	\begin{tabular}{|M{1cm}|M{2cm}|M{2cm}|M{2cm}| M{2cm}|M{2cm}|}
	\hline
	\multirow{2}{*}{N} & $t$ & $d_{1}$ & $d_{2}$ & $d_{\text{ср}}$ & $HV$ \\
	\cline{2-6}
	& c &\multicolumn{3}{c|}{мкм} & -- \\
	\hline
	1 & 5 & 51.0 & 50.0 & 50.5 & 145.4  \\
	
	2 & 20 & 52.8 & 51.9 & 52.4 & 135.3 \\
	3 & 	40 & 52.3 & 51.8 & 52.8 & 136.9 \\
	4 & 60 & 52.5 & 51.5 & 52.0 &  137.1 \\
	\hline
	\end{tabular}
     \caption{Исследование микротвердости медного образца при различном времени идентирования и нагрузке 2Н}
     	\label{Cop_time}
     \end{table}
     
     
     \begin{table}[h]
     \centering
     	\begin{tabular}{|M{3cm}|M{3cm}|M{3cm}|}
     	\hline
     	& Сталь & Алюминий \\
     	\hline
      	$\overline{X}$  & 167.0 & 133.9 \\
     	$\sigma = \sqrt{\frac{\sum(X_{i} - X_{ср})}{n - 1}}$ & 16.6 & 4.8 \\
     	$S_{x} = \frac{\sigma}{\sqrt{n}}$                    & 5.3 & 1.5 \\
     	$\Delta X_{случ} = S_{X}\cdot t_{\alpha_{N}}$          & 9.5 & 2.7 \\
     	$\frac{\Delta X}{\overline{X}}\cdot 100\%$           & 5.7$\%$ & 2.7$\%$ \\
     	$W = \frac{\sigma}{\overline{X}}$                    & 9.9$\%$ & 3.6$\%$ \\
     	                    
     	\hline
     	\end{tabular}
     \end{table}
     
    \newpage
    
    \section{Экспериментальная установка}
    

    \newpage
    
    
\end{document}